\chapter{Fundamentação Teórica}
\label{ch:fundamentacao}
\par Neste capítulo serão fundamentados os conhecimentos b\'asicos para o entendimento do trabalho e as vulnerabilidades e possíveis ataques que podem ser considerados ameaças comuns em redes de \emph{IoT} que utilizam a topologia \emph{mesh} com a comunica\c{c}\~ao sem fio. Para identificar as vulnerabilidades, analisaremos uma rede que tem como c\'odigo fonte dos microcontroladores um exemplo de \emph{LWMesh} e por meio de um julgamento de custo e impactos, será atribu\'ido um nível para a vulnerabilidade.
A seguir serão apresentadas vulnerabilidades e falhas que são importantes para o desenvolvimento do trabalho.

\section{Redes de IoT}
\par A IoT tem dois meios para transmitir as informa\c{c}\~oes, o meio com fio e sem fio, portanto neste trabalho ser\'a abordado as transmiss\~oes sem fio \emph{(wireless)}.
\subsection{\emph{Standard} IEEE 802.15.4}
\par Este \'e um \emph{standard} criado por \citeonline{8021542006} que define os protocolos de comunica\c{c}\~ao e a interconex\~ao dos dispositivos via r\'adio. Alguns protocolos que utilizam este \emph{standard} por exemplo s\~ao: ZigBee, MiWi e LWMesh.
\par Os protocolos criados em cima deste \emph{standard} possuem as seguintes caracter\'isticas:
\begin{itemize}
    \item Topologias: Estrela ou malha;
    \item Endere\c{c}amento de 16 bits a 64 bits;
    \item Baixo consumo de energia;
    \item Indicador de Qualidade de \emph{Link} (LQI);
    \item Podem operar em 16 canais na frequ\^encia de 2.4GHz, 30 canais na frequ\^encia de 915MHz e 3 canais na frequ\^encia de 868MHz;
\end{itemize}

\par Divide-se em camadas \emph{(layers)} baseado no modelo OSI para facilitar a compreens\~ao, cont\'em a camada f\'isica (PHY) que conta com o transceptor de r\'adio frequ\^encia, um mecanismo de controle e uma camada de meio de controle de acesso (MAC) que \'e respons\'avel pelo servi\c{c}o de transmiss\~ao e recep\c{c}\~ao de dados, e fornece meios de implementa\c{c}\~ao de mecanismos de seguran\c{c}a apropriados na aplica\c{c}\~ao.

\section{Conceitos de Seguran\c{c}a}
\par Segundo \citeonline{pfleeger2002security} o termo ``seguran\c{c}a'' \'e utilizado de diversas maneiras no dia a dia, entretanto todos esses termos tem seus significados de acordo com o contexto utilizado. Na computa\c{c}\~ao o termo seguran\c{c}a visa tratar tr\^es aspectos importantes: confidencialidade, integridade e disponibilidade.
\begin{itemize}
    \item Confidencialidade: Assegura que as informa\c{c}\~oes ser\~ao acessadas apenas por pessoas autorizadas. Tamb\'em pode ser chamado de privacidade.

    \item Integridade: Significa que os dados podem ser apenas modificadas por pessoas ou por meios autenticados. Neste contexto, a modifica\c{c}\~ao inclui escrita, altera\c{c}\~ao, exclus\~ao e cria\c{c}\~ao.

    \item Disponibilidade: Significa que as informa\c{c}\~oes podem ser apenas acessadas por pessoas autorizadas ou em hor\'arios permitidos. O oposto de disponibilidade que ser\'a citado ao decorrer do trabalho \'e nega\c{c}\~ao de servi\c{c}o.
\end{itemize}

\par A seguir ser\'a conceituado os ataques que foram citados no desenvolvimento deste trabalho.

\begin{table}[ht]
\centering
\caption{Tabela de Ataques}
\label{Tabela-de-Ataques}
\begin{tabular}{|l|l|l|}
\hline
\textbf{Camada}         & \textbf{Ataques}         & \textbf{Defesas Conhecidas}    \\ \hline
\multirow{2}{*}{Física} & \emph{Jamming}           & N\~ao Aplic\'avel              \\ \cline{2-3} 
                        & Escuta Passiva           & Criptografia                   \\ \hline
\multirow{2}{*}{Link}   & Colisão de Pacotes       & Implementa\c{c}\~oes no codigo \\ \cline{2-3} 
                        & Exaustão                 & Limitação de Envios            \\ \hline
Rede e Roteamento       & \emph{Man in the Middle} & Autorização e Monitoramento    \\ \hline
Transporte              & \emph{Flooding}          & Autenticação                   \\ \hline
\end{tabular}
\end{table}

\section{Ataque de Negação de Serviço}
\label{sec:2dos}
\par Ataque de negação de serviço, ou \emph{Denial Of Service (DoS)} segundo \citeonline{wood2002} são tipos de ataques que podem comprometer a rede de forma crítica por meio de perturbação na rede ou até mesmo na invalidação da mesma. Esse tipo de vulnerabilidade pode ser causada por exploração de falhas de hardware, bugs no software, exaustão dos recursos dos nós ou até mesmo condições do ambiente.

\section{\emph{Jamming}}
\label{sec:2jammer}
\par Muitos dispositivos são dependentes de redes de comunicação sem fio, por\'em o bloqueio desses sinais tornando-os indispon\'iveis \'e denominado \emph{Jamming}. \'E uma grande ameaça para redes sem fio e entender a complexidade deste tipo de ataque e suas contra medidas é de suma importância, j\'a que este é um ataque contra a disponibilidade do sinal e há diversos tipos de \emph{jammers} \cite{wilhelm2011a}.

\par Como explicado por \citeonline{alturkostani2015} há diferentes tipos de \emph{jammers}, e entre eles os que mais se destacam s\~ao: \emph{jammers} constantes por ser o mais utilizado e os \emph{jammers} inteligente, que, por analisar o tráfego, alvejam pacotes específicos e emitem ruídos para corromper o conteúdo dos mesmos.

\begin{figure}[ht]
	\caption{\emph{Jammer} Constante}
	\centering
		\includegraphics[width=\textwidth,height=\textheight, keepaspectratio]{figuras/jammerContinuo}
	\fonte{Elaborada pelo autor.}
\end{figure}

\section{Escuta Passiva}
\label{sec:2sniffing}
\par Neste tipo de ataque, o atacante monitora a rede sem fio utilizando um software em busca de informações que são trafegadas através de uma antena direcional no modo monitor e no canal onde a rede est\'a operando. Estes tipos de ataques não podem ser facilmente detidos por medidas de seguranças de \emph{software}, como proteger os dados trafegados através de criptografia, entretanto, este \'e um modo de mitigar. Assumindo uma rede sem a proteção de criptografia em seus pacotes enviados pela rede, o atacante ganha informações importantes utilizando esse ataque. O invasor consegue analisar os pacotes e descobrir seu remetente, destinatário, tamanho e tempo de transmissão. O impacto deste ataque não é um risco apenas à privacidade, mas sim é uma precondição para ataques mais nocivos \cite{welch2003a} \cite{yuan2008}.

\par As \emph{WMNs} s\~ao suscetíveis à escuta interna por meio de seus n\'os intermediários, onde um n\'o malicioso pode manter uma cópia de todas as informações encaminhadas para ele sem o conhecimento dos outros n\'os da rede, entretanto, este ataque não interfere diretamente na funcionalidade da rede, apenas compromete a privacidade e integridade dos dados. A criptografia dos dados é implementada utilizando uma chave para proteger os dados e manter sua integridade e privacidade \cite{sen2013}.

% \section{\emph{MAC Spoofing}}
% \label{sec:2macspoofing}
% \par O \emph{MAC Spoofing} refere-se a alteração do endereço \emph{MAC} \emph{(Media Access Control)} de uma interface de rede. Cada interface de rede possuí um endereço \emph{MAC} único que pode ser alterado via hardware ou software. Alguém mal intencionado pode utilizar a técnica de \emph{MAC Spoofing} para tomar a identidade de um dispositivo autenticado e entrar em redes com aquela identidade. A técnica consiste em primeiramente for\c{c}ar a desconex\~ao ou esperar o computador ou dispositivo da rede se desconectar, tomando assim sua identidade, tornando poss\'ivel a intrus\~ao na rede sem ser detectado como intruso. O \emph{MAC Spoofing} pode ser utilizado em ataques \emph{DoS}, por exemplo, em \emph{Authentication flood DoS Attack} o invasor envia \emph{frames} de requisição de associação composta de endereços \emph{MAC} randômicos na tentativa de sobrecarregar o \emph{AP} com diversas requisições \cite{yuan2008} \cite{sen2013}. 

\section{\emph{Man in The Middle}}
\label{sec:2mitm}
\par Um ataque MITM \emph{(Man in The Middle)} segundo \citeonline{studyMITM} tem como conceito fundamental entrar no meio de uma conex\~ao tornando-se uma ponte entre o cliente e o destinat\'ario, portanto o mesmo envia os pacotes normalmente para o destinat\'ario e pode realizar fun\c{c}\~oes ilegais na rede, abrindo a possibilidade de diversos ataques.

\begin{figure}[ht]
	\caption{\emph{Man in The Middle}}
	\centering
		\includegraphics[width=\textwidth, height=\textheight, keepaspectratio]{figuras/mitmExample}
	\fonte{Elaborada pelo autor.}
\end{figure}

\par A seguir ser\'a explicado o fluxo de como o ataque ocorre na teoria.
\begin{enumerate}
    \item A comunica\c{c}\~ao entre o cliente e o ponto de acesso \'e monitorada para conseguir as informa\c{c}\~oes necess\'arias como o \emph{SSID} e o canal do mesmo.
    \item O atacante inicia um ponto de acesso com as mesmas informa\c{c}\~oes do ponto de acesso aut\^entico.
    \item O acesso entre o cliente e o ponto de acesso \'e interrompido por um jamming ou alguma t\'ecnica que cause a indisponibilidade do AP.
    \item O cliente tenta se autenticar com o mesmo \emph{SSID} da rede aut\^entica, entretanto n\~ao encontra a rede original e se conecta na rede do atacante que possui as mesmas informa\c{c}\~oes.
    \item Os pacotes enviados do cliente s\~ao interceptados pelo atacante e replicados para o ponto de acesso aut\^entico.
\end{enumerate}

\subsection{\emph{Wormhole Attack}}
\par O \emph{wormhole attack} ou ataque de tunelamento é um dos mais poderosos e severos ataques em \emph{WMNs}. Os métodos tradicionais de segurança, como criptografia e \emph{digital signature} podem prevenir o comprometimento, integridade e privacidade dos pacotes que estão sendo trafegados, porém o \emph{wormhole attack} é transparente para esses métodos \cite{zhou2012}.

\par O \emph{wormhole} coloca o atacante em uma posição muito privilegiada, ele é capaz de explorar a rede e realizar diversos ataques, permitindo que o invasor obtenha acesso sem autorização, ou rompa do roteamento, ou até mesmo causar indisponibilidade na rede \cite{hu2003}.

\begin{figure}[ht]
	\caption{\emph{Wormhole Attack}}
	\centering
		\includegraphics[width=\textwidth, height=\textheight, keepaspectratio]{figuras/wormholeAttack}
	\fonte{Elaborada pelo autor.}
\end{figure}

\par O invasor pode encaminhar as mensagens recebidas através de um meio com baixa latência e o replica em uma parte diferente da rede. Geralmente envolvem dois n\'os maliciosos distantes entre si transmitindo pacotes ao longo de um meio distinto disponível apenas para os n\'os maliciosos. Em alguns casos o atacante consegue romper totalmente o roteamento através de um \emph{wormhole} bem implementado, e isso resulta na possibilidade de convencer os n\'os de que a rede está funcionando normalmente e que seriam múltiplos saltos a partir de uma estação base, sendo que eles estão apenas um ou dois saltos de distância através do \emph{wormhole}. O mesmo pode ser utilizado para criação de \emph{sinkholes} pela potencial atratividade do roteamento de um n\'o comprometido ou malicioso \cite{karlof2003}.

\subsection{\emph{Sinkhole Attack}}
\par Estes ataques comprometem a aparência de um n\'o fazendo com que se torne atrativo para os n\'os vizinhos. Fazem isto respeitando o algoritmo de roteamento da rede com o principal objetivo de prevenir que a estação principal receba o dado completo e correto \cite{ngai2006}. O invasor pode encaminhar as informações para a estação principal através de um roteamento de alta qualidade fazendo com que os protocolos que visam a confiabilidade ou a baixa latência prefiram aquele n\'o comprometido para enviar as informações para a estação principal. Há a possibilidade do invasor utilizar um \emph{wormhole} ao invés de um roteamento de alta qualidade. \cite{salehi2013}

\begin{figure}[ht]
	\caption{\emph{Sinkhole Attack}; (a) n\'o comprometido; (b) Estação principal.}
	\centering
		\includegraphics[width=\textwidth, height=\textheight, keepaspectratio]{figuras/sinkholeAttack}
	\fonte{Elaborada pelo autor.}
\end{figure}

\par O \emph{sinkhole attack} pode abrir oportunidades para diversos ataques diferentes, entre eles o encaminhamento seletivo das informações e o \emph{blackhole}. O mesmo pode ser lançado realizando dois passos: primeiro, o n\'o comprometido identifica seus vizinhos que tem as menores distâncias em relação a estação principal, e em seguida envia informações de que o n\'o comprometido tem a menor distância. Realizando os passos anteriores, o atacante consegue mudar o mecanismo de roteamento e o n\'o adversário passa a funcionar como um \emph{sinkhole}. Após um lançamento de sucesso, certificando que o tráfego da área selecionada passe pelo n\'o comprometido, o invasor pode suprimir ou modificar as mensagens originadas de qualquer n\'o na área \cite{qi2012}.
